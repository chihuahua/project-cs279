\section{Introduction}

%how to cite
%These guys wrote awesome stuff~\cite{tohidi06:getting,tohidi06:user}

%This section should address:
%Motivation: What problem will your work address? Why is it important?
%Approach: What's your insight? How are you going to go after the problem?
%Contribution: Lasting knowledge that will result from your work.

We are attempting to learn how much first impressions might impact the actual adoption of an Android game or productivity application. Mainly, we will examine what factors relevant to first impressions may contribute to the success of such apps. We plan to measure success through popularity (i.e. downloads, ratings). We hope to determine a good model for predicting application successes by examining observable measures for success of mobile applications on the Android Store.

Currently, app developers take substantial measures to gauge the success of their apps prior to release. Many release a beta version of their app to a restricted community before releasing the app to the public. This beta test allows the developers to not only debug the app, but also obtain an estimate of how much interest their apps will garner. Conducting a meaningful beta test requires a reliable and closed community of trustworthy individuals. Many large firms have this community on hand. For instance, before Google released its iOS application for its Maps service, it first released it to its thousands of employees. Google then got feedback from these employees as well as statistics such as how long individuals stayed on the app and how often they returned. However, small developers do not have such a substantial community of individuals at their disposal for beta testing. Often, these developers just publish their app and monitor metrics over time, adjusting along the way. Our tool would help these developers by offering valuable predictors of app success. 

Additionally, developers often test specific facets of their app on their own through ads and crowdsourcing services. To test how well an icon performs relative to viable alternatives, app developers sometimes release several different ads on social media sites such as Facebook. Each of these ads would showcase a different candidate icon for an app. A common measure for how well an icon performs in these cases is click-through rate. Similar ideas are often applied to tag lines and descriptions of apps. Additionally, services such as Ikonica offer developers the opportunity to send potential icons to a crowd of online individuals and collect their feedback. Our tool would not only formalize such processes that test different facets of an app in a single tool, but also offer a cheaper method for gauging app success through utilizing the common sense of Turkers or potentially other crowds.

We posit that we can create a way of predicting the success of a game or productivity application without having the developer release an app and monitor metrics in the wild for a duration. We base our tool on the premise that certain factors such as how fun the page of an app appears and how much a viewers trust that page or the app’s icon play a pivotal role in determining the game’s success.

Granted, many factors go into the success of game and productivity apps. Games, for instance, must be easily graspable at first, for instance. They must also become harder at a reasonable pace to keep the user engaged. Productivity apps must effectively meet concrete needs. We will be focusing on qualities pertaining to the page of an app, which matters in making strong first impressions.

We plan to leverage the common sense of crowds to gauge these facets and use this data to offer developers an estimate of the success of an app. We plan to develop a statistical model, perhaps based on regression from data gained from the crowd. To obtain data, we plan to scrape the names, icons, and descriptions of free game and productivity apps, so price does not bias one game over another. We plan to analyze Android apps since Android provides downloads statistics for apps albeit in a discretized manner.

Afterwards, we plan to develop a questionnaire that asks subjects to rate apps based on the different qualities on a Likert scale. These qualities could include how much subjects trust app pages, whether the pages feel fun, or how clear does the page describe the app’s purposes and functionalities. We plan to gamify this process by challenging subjects to test their entrepreneurial gut. We also plan to have Turkers be subjects. We plan to scrape about 500 recent apps and will likely need several thousand, perhaps 5000 subjects.

To prevent biases, we should also ask users if they had seen the current app before asking them questions about it. We will likely discount users who have seen the app. Currently, we plan on just performing the analysis on recently loaded apps. We might also scrap apps that had been on the market for a while and compare the results between these two groups. We also plan on just scraping free apps so that price is not a confounding variable in our experiment. We should also ask subjects about their smartphone habits - are they familiar with smartphones and apps on them? Or have they never used a smart phone?

After obtaining this data, we plan to perform a regression to determine how much each quality matters in determining the success of game apps. We then plan to use this regression result as a tool to predict the success of other apps. We also plan to talk to Katharina Reinecke about how else we can interpret our results.

Our goal is to establish the link between the way people feel about given, observable measurements and the application's actual success, thus delivering a novel approach to predicting the success of mobile applications. We also plan to use this intuition to create a tool that allows users to cheaply predict the success of an application. Beyond software, we hope to determine what factors of the page of a game and productivity application that truly matter in determining the success of the app. This lasting piece of knowledge should benefit developers of games and productivity apps.
 
In other words, we intend to uncover the discrete, concrete elements that create subtle psychological influences in users and that propel users to download and use certain  application versus others.  Marketing is a very powerful tool; therefore, through uncovering the words, images, perceptions, and feelings that marketers have historically conveyed through their advertisements in the App Store and dissecting resulting users’ reactions to these subtleties, we can determine the elements that separate a mediocre from a successful marketing campaign.
