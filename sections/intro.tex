\section{Introduction}

%what problem will your work address? why is it important?
%explain why first impressions matter: what first impressions lead users to do
%how people make their judgments (first impressions vs rational evaluation of a product)
%how those previous things make it reasonable to think our hypothesis
%what problem will your work address? why is it important?

It has been well established that first impressions matter. From interview preparations to website aesthetics, people understand the power of an image. Moreover, people can form opinions within the first few seconds of viewing a particular product and use that impression to determine the potential for future company loyalties. That is the reason we decided to examine if first impressions mattered in the context of mobile applications (apps).  In fact, knowledge of this concept has led independent mobile application developers to attempt to solicit feedback on specific features of their application through advertisements and crowdsourcing services. Nevertheless, such services can be costly and the process, being unformalized, does not offer the developer a holistic view of the potential success of their application.  It simply offers them an idea of users' opinions regarding a feature in isolation.  Therefore, the ability to forecast the level of success that a future moblile application can achieve ahead of deployment can be an extremely valuable tool\cite{tohidi06:getting}. Although, the method of predicting success can vary, we wanted to determine if crowds can be leveraged to predict app success with trivial app descriptors. \\  

%Existing approaches involve deploying applications and potentially failing. To avoid the time-consuming and expensive process of deploying an application that might fail, mobile application developers, for example, take many measures to gauge the success of their applications prior to release.  The first may be the release of a beta version of their application to a restricted community before the release of that app to the public.  This allows them to estimate future interest. However, conducting a meaningful beta test requires a reliable and closed community of trustworthy individuals. Although many large firms have such resources, independent mobile application developers have to deploy their applications and modify them later \cite{website06:mobile}.  They learn through trial-and-error how to make their application more appealing.  Our approach has the potential to improve the aforementioned design process by helping designers get the right design prior to deployment \cite{tohidi06:getting}.

The typical existing approach involves deploying an application and potentially failing. To avoid the time-consuming and expensive process of deploying an application that might fail, mobile application developers, for example, take many measures to gauge the success of their applications prior to release.  One of these methods includes the release of a beta version of the application to a restricted community before releasing the app to the public: This assists them in developing a more accurate estimation of future interests \cite{betatest}. However, conducting a meaningful beta test requires a reliable and closed community of users. Although many large firms have such resources, independent mobile application developers are reduced to deploying their application and repeatedly modifying it in hopes that, through trial-and-error, the application might become more successful \cite{betatest}.

%what's your insight? how are you going to go after the problem?
For our approach, we will use Android mobile games to test our experiment. We choose to use mobile games for our experiment as the data is widely available. As users' first impressions of mobile applications are most often made through the application store, our study will simulate the features found on such pages. We seek to establish this link by conducting a series of online studies. Success will be measured by an app's popularity, represented by the number downloads it obtains. Measurements, which we believe crowds can use to accurately predict success, include the icon, slogan, screenshot, and description of the app. Our studies will also include a demographics survey in order to better assess the mechanisms underlying the connection. For instance, women and men may have systematic preferences for certain designs over others and equate future successes to apps that they perceive to be more trustworthy or fun.  These same apps that they predict to be more trustworthy or fun are most likely the apps that they will tend to gravitate towards.

%lasting knowledge that will result from work
%% == SIERRA: please figure out how to properly format bullet points per SIGCHI's formatting ==%%
We make the following contributions:
\begin{itemize}
\item We will attempt to establish a link between the way users feel about a given application measurement and the application's actual success.
\item We hope to deliver a novel approach to predicting the success of mobile applications.
\end{itemize}

