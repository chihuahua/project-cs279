\section{Introduction}

%% == SIERRA: please figure out how to properly format bullet points for contributions below per SIGCHI's formatting ==%%
%% == SIERRA: please add sources where you see cite{need source}. some sources may be in the related work, some may need to be added ==%%

%what problem will your work address? why is it important?
%explain why first impressions matter: what first impressions lead users to do
%how people make their judgments (first impressions vs rational evaluation of a product)
%how those previous things make it reasonable to think our hypothesis
%what problem will your work address? why is it important?
%what's your insight? how are you going to go after the problem?
%lasting knowledge that will result from work

It has long been a dream of application creators to know whether a given product will be successful or not ahead of its deployment. Knowing whether a given website or application will be a success prior to releasing a product out into the wild could potentially save designers billions of dollars~\cite{needsource}. But the question still lies unanswered: how do you predict the success of an application without first deploying it?

The existing approach involves deploying an application and potentially failing. To avoid the time-consuming and expensive process of deploying an application that might fail, mobile application developers, for example, take many measures to gauge the success of their applications prior to release, one of which is releasing a beta version of their application to a restricted community before releasing the app to the public in order to estimate future interest~\cite{needsource}. However, conducting a meaningful beta test requires a reliable and closed community of users. Although many large firms have such resources~\cite{needsource}, independent mobile application developers have to just deploy their application and modify it, if possible, so that through trial-and-error the application might become more successful.

Further, it has been well-established that first impressions matter. From interviews to website aesthetics, users form an opinion within the first few seconds of viewing a given stimuli~\cite{needsource}. Thus far, research on first impressions has only been examined within the context of websites~\cite{needsource}, we seek to further first impression insights by exploring them within the context of mobile applications.

Since designers know that first impressions matter, independent mobile application developers have tried obtaining feedback on specific features of their application through advertisements and crowdsourcing services~\cite{needsource}. Nevertheless, such services can be costly, and their feedback process is not formalized or offers the developer a holistic view of the potential success of their application, simply of the feature in isolation.

Our goal is to combine the insights on product success and first impressions by predicting the success of mobile applications based on users' first impressions. Being able to predict the success of mobile applications  based on observable measures will allow us to establish a link between the first impressions of objective measures and the application’s eventual success. In addition, our approach has the potential to improve the existing design process by helping designers get the right design prior to deployment.

For our approach, we will use Android mobile games to test during our experiment. We chose to use mobile games for our experiment as the data is widely available. As users' first impressions of mobile applications are most often made through the application store, our study will simulate the layout found on such pages to replicate conditions and streamline variables. We seek to conduct a series of online studies, where the success of an application is measured by its popularity, represented by the number of its downloads. Potential measurements to predict success used include icon, slogan, screenshots (aesthetic appeal), and description. Our studies will also include a demographics survey in order to better assess the mechanisms underlying the connection. For instance, people may think a given application will be successful because they perceive it as more trustworthy or fun.

We make the following contributions: \\ \\
- We establish a link between the way users feel about a given application measurement and the application's actual success.\\
- We deliver a novel approach to predicting the success of mobile applications.
