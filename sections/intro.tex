\section{Introduction}

%what problem will your work address? why is it important?
It has long been the goal of many to be able to know ahead of deployment whether a given product will be successful or not. Knowing whether a product will be a success ahead of deployment could save designers billions of dollars.~\cite{tohidi06:getting} But the question still likes, how do you predict the success of an application?

The existing approach involves deploying an application and potentially failing. To avoid the time-consuming and expensive process of deploying an application that might fail, mobile application developers, for example, take many measures to gauge the success of their applications prior to release, one of which is releasing a beta version of their application to a restricted community before releasing the app to the public in order to estimate future interest. However, conducting a meaningful beta test requires a reliable and closed community of trustworthy individuals. Although many large firms have such resources (google article source!), independent mobile application developers have to rather deploy their application and modify, if possible, so that through trial-and-error the application might become more successful.

Our approach has the potential to improve the aforementioned design process by helping designers get the right design prior to deployment.

%explain why first impressions matter: what first impressions lead users to do
%how people make their judgments (first impressions vs rational evaluation of a product)
%how those previous things make it reasonable to think our hypothesis
%what problem will your work address? why is it important?
It has been well-established that first impressions matter. From interviews to website aesthetics, users form an opinion within the first few seconds of viewing a given stimuli. Thus far, first impressions have been used in the context of websites.

Since designers know that first impressions matter, independent mobile application developers have tried to obtaining feedback on specific features of their application through advertisements and crowdsourcing services (sources! Ikonica!). Nevertheless, such services can be costly and the process is not formalized and does not offer the developer a holistic view of the potential success of their application, simply of the feature in isolation.

Our goal is to combine the insights on product success and first impressions by predicting the success of mobile applications based on users' first impressions. Being able to predict the success of mobile applications  based on observable measures will allow us to establish a link between the first impressions of objective measures and the application’s eventual success.

%what's your insight? how are you going to go after the problem?
For our approach, we will use Android mobile games to test during our experiment. We chose to use mobile games for our experiment as the data was widely available. As users' first impressions of mobile applications are most often made through the application store, our study will simulate the features found on such pages.

We seek to establish this link and predicting the success of a mobile application through users' first impressions by conducting a series of online studies. The success of an application is measured by its popularity, represented by the number of its downloads. Potential measurements to predict success used will be icon, slogan, screenshots (aesthetic appeal), and description.

Our studies will also include a demographics survey in order to better assess the mechanisms underlying the connection. For instance, people may think a given application will be successful because they perceive it as more trustworthy or fun.

%lasting knowledge that will result from work
%% == SIERRA: please figure out how to properly format bullet points per SIGCHI's formatting ==%%
We make the following contributions: \\ \\
- We establish a link between the way users feel about a given application measurement and the application's actual success.\\
- We deliver a novel approach to predicting the success of mobile applications.
