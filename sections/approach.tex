\section{Approach}

To tackle this problem, we scrape 500 recent mobile game applications from the Android market. We recruit 250 participants through social media to perform 10 rounds. For each round, a subject will be asked to predict which of two apps is more successful. For the first part of this experiment, subjects are only shown apps' names and icons. For the second part, subjects are shown the first screenshot of apps as well. Subjects must make selections within ten seconds. The survey is powered by Django and a MySQL database. \\

We make the survey more appealing by treating it as a test of subjects' entrepreneurial guts. After users answer which application they think is more successful, our system will show them the real answer and ask them the reason they chose a certain option. We hope that the gamification of this experiment will lead to more users willing to participate in our survey and more experimental data to analyze. Subjects will see a score denoting what proportion of trials they guessed correctly. We will later reveal to users how other users performed on average, creating a competitive environment.\\

Afterwards, we will use an algorithm to rank the applications based on which ones users found to be more successful. We will see how well these rankings correlate with how successful the apps actually are in terms of download counts.

%\begin{itemize}
%HOW GOOD IS YOUR ENTREPRENEURIAL GUT?
%agreement to statement of privacy
%two name of apps + their icons
%option to opt-out if they've already seen app
%which one do you think is more successful? (select one)
%why do you think this one is more successful? (free-form comment)
%result of whether they were right or wrong + score + try again!

%\item Control for difference in application cost by only using free applications
%\item Obtain approximately 500 mobile applications and 5000 participants 
%\item Utilize the application store: Preference towards Androids since they publicly provide download statistics
%\item Use samples from games and productivity applications
%\item Incentivize users by making the study fun to complete and additionally taking advantage of users from Mechanical Turk to cover the remaining numbers required.
%\item Performing a quick survey for the first five users
%\item Display applications in different orders to weed out systematic preferences based on order.
%\item Utilizing apps with roughly the same distribution of dates of release
%\item Separating applications into two categories: (1) relatively new to the market, so users have not seen them before, and (2) been in the market for a while
%\item Displaying a mock-up of the application store listing (all must be the same except for the variable being tested)
%\item Scaping the the names, icons, and descriptions of the aforementioned applications from the Android store
%\end{itemize}
%
%We plan to conduct a survey across crowds that gauges their views on various aspects of web pages of applications. We control for differences in application cost by only examining free applications.
%We will scrape about 500 applications and get about 5000 participants to do the survey (Our sample must be big to offset random error.).
%We plan to use Android apps since its application marketplace publicly provides download statistics, albeit in discretized increments.
%We plan to examine a specific subset of apps: games + productivity app (The use of games is discretionary, so we must also examine something else that people need to use). \\
%
%We will incentivize users by making the study fun to complete. If not enough users complete the study in this way, we will proceed to obtain some users via Mechanical Turk.
%setup: quick survey + first batch of 5
%We plan to show applications in different orders to different people (to eliminate systematic biases).
%We will try to have roughly the same distribution of dates for when the applications were released.
%We plan to separate applications into 2 categories: (1) relatively new to the market, so users have not seen them before, and (2) been in the market for a while
%show mock-up of application store listing (all must be the same except for the variable being tested) if Android releases information on older applications.
%To obtain data, we plan to scrape the names, icons, and descriptions of the aforementioned applications.
