\section{Approach}

In order to tackle this research problem, we scraped 500 mobile game applications from the Android market with the goal to create two online surveys. The first one focused on the title and icon of two applications whereas the second one focused on the last two variables as well as a screenshot of the application. Thus, we conducted a between subjects study since different subjects took the first and second surveys.

For each survey trial, a subject was asked to compare two factors (or three in the case of the second survey) between two distinct applications. The applications were presented to users in a random basis. More specifically, to reduce the chance of users seeing the same application multiple times, client-side randomization was used. Further, the survey was powered by Django and a MySQL database.

Participants were only shown the variables of the apps for fifteen seconds as we wanted to obtain their very first impression. During the trials, users were asked to briefly explain their selections 20\% of the time. In additon, we made our trials more appealing by encouraging users to test their business savvy and attempt to improve their score. After 10 trials, participants received a score denoting what proportion of trials they guessed correctly. From this point on, after every trial, subjects were informed what was their updated score.