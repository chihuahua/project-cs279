\section{Results and Analysis}

%The data should be rich enough so that you can reflect on which aspects of your project are working and which arent
%This should be a brief summary of the results and what they mean for your continuing progress

We found that screenshots did not significantly increase the accuracy of subjects' predictions of apps' relative successes. On average, the 135 individuals exposed to apps' first screenshots predicted 46.1\% of the rounds correctly. For the 29 individuals exposed to just the names and icons of apps, the figure was 42.6\%. The p-value for two-tailed t-test assuming equal variances was 0.324, so the difference between the two groups was not significant. The 50 females and 82 males who were only exposed to apps' names and icons also did not exhibit significant differences between each other in their abilities to predict app success. \\

Unfortunately, since the average overall accuracy is under 50\%, we cannot build a system for gauging the success of apps based on names, icons, and screenshots alone. It seems that factors beyond these features of an app matter. Perhaps the quality of the app or marketing strategies far trump the features on an app page in terms of determining how many downloads apps receive. Further research can build on these results by finding out factors that do matter for game apps. Perhaps a study that examines the success of apps just a few hours after their releases would more effectively determine the contributions of icons and screenshots towards the success of apps. In those cases, far fewer users would have downloaded the apps, so the effect of the first impression that app pages offer would be more pronounced. This study is hard to conduct, however, since most market places for apps do not reveal the specific times of release of apps.

We also performed an ordinal logistic regression on how often individuals downloaded and used apps on their phones. Very interestingly, we discovered that individuals that downloaded and used apps more often did not have significantly higher accuracies. From this result, we conclude that individuals of the general populace are not experts at predicting which apps will be downloaded more often than others. Those individuals who use smartphone apps daily are no exception. Any tools that try to gauge the success of apps based on these features of the app page alone will likely be ineffective. \\
