\section{Related Work}

%related work syntheses focus on conceptual points
%highlight how different prior projects addressed similar issues (sometimes highlighting similarities, sometimes bringing together complementary evidence, sometimes pointing out lack of consistency across different publications)
%check out B. J. Fogg at Stanford, might have done some relevant research

%% == SIERRA: who is this paper by? ==%%
%% == SIERRA: can you make sure each of these related works have its properly formatted entry on our .bib document? ==%%
Quantifying Visual Preferences around the World
This paper analyzes how people from different demographics react to how colorful web pages are. The experiment collected ratings of visual appeal from about forty thousand subjects on 430 web pages of varying complexities and colorfulness. The qualities that subjects performed ratings on include trustworthiness and  of the web page. The experiment found that certain demographics such as Russians and Fins did not find colorful web pages as appealing as did demographics such as Macedonians.
This paper relates to our work since we too seek to find factors that heighten appeal. Specifically, we seek factors that maximize the appeal of pages for game and productivity apps on the Android platform. Icon color could be one of the factors we analyze. We could also borrow off of many ideas from the methodology. For instance, the researchers motivated individuals to do their test by comparing their results to other individuals at large. Our method for performing regression might be more complicated though since we analyze more factors than just one (say color saturation of a page).
Overall, I enjoyed reading this paper. The results showed that most people preferred a moderate level of color saturation in web pages. I feel that some of the conclusions about specific demographics of people and their color preferences succumbed to a small amount of response bias though. In some countries, only a small portion of residents have access to the internet.

The Business of iPhone App Development: Making and Marketing Apps that Succeed
By Dave Woolridge and Michael Schneider (2010)
This book offered us some keen insights on how app developers currently gauge the success of their apps both after and before release. For instance, many services out there such as Mobclix provide rankings of apps based on download figures. App developers also read over the reviews of competitors to determine how well their services will be received.
App developers also emphasize first impression a lot according to this book. For instance, they value the layout of their icons. Developers should also value communicating a consistent message to the user. From this book, we determined several factors that could heavily influence first impressions of app pages. I wished that the book discussed how crowd sourcing could help app developers gauge app success though. This book was written in 2010, and gaining knowledge from crowds was not as substantial of a concept back then.

Predicting Users's First Impressions of Website Aesthetics With a Quantification of Perceived Visual Complexity and Colorfulness
This paper examined how different groups of people perceived web pages of varying complexity and color saturations. The experimenters asked 548 participants to rate 450 different websites on a number of metrics. What intrigued me most about this paper was how it used quantitative measures to ascertain such soft qualities as page complexity and colorfulness. Perceived colorfulness even depended on the context of the colors. The experimenters nicely used the sum of the average and the standard deviation of the saturations to measure perceived colorfulness.
In our study, we will have to take similar measures. We are also trying to quantitatively gauge the effects of such soft qualities as trustworthiness and fun-ness of an app's page.

Attention web designers: You have 50 milliseconds to make a good first impression
This paper examined how viewers of web pages really make their first impressions about a website within a very short amount of time. The paper emphasizes how aesthetics is often neglected in current studies on emotion and design. Apparently, emotional responses can be triggered much more quickly than rational ones. Humans are quick to assign words such as clean, symmetric, and dark to images. This article hence directly relates to our current studies since it discusses how first impressions can significantly impact people's emotional response to an app. Hence, we should ask users to rate their emotional responses to various apps and/or their icons. However, I feel that this paper also somewhat understates the importance of functionality. I wish it could further examine how important this emotional response is.~\cite{tohidi06:getting}

App Empire: Make Money, Have a Life, and Let Technology Work for You
By Chad Mureta (2012)
Since marketing plays such an essential role in the research we intend to pursue, we thought it best to examine literature related to mobile applications marketing research. In this book, Mureta highlights key insights regarding mobile application design and marketing which separate successful apps from non-successful ones.  Using raw, forthright diction, Mureta acts as sort of a personal mentor to the reader, using words that convey a sort of familial bond that is greatly disarming, psychologically enticing, but more importantly packed with valuable tips for fellow application entrepreneurs.  It is this direct, candid advice that we seek to capture.  Furthermore, with his pointed market overviews that focus on mobile application usage and financial statistics and his instructional section on "Sex App-eal" in which he maintains a discourse on the importance of icons, titles, descriptions, screen shots, keywords,and categories, Mureta's work will endow us with the knowledge and references we need to augment our understanding of mobile application design and develop better, more informed hypotheses regarding what makes a mobile application successful. Finally, the reliability of Mureta's claims are backed by his own success in the mobile application industry.~\cite{mureta12:app}

Mobile Marketing Research Priorities: Roadmap to Engaging the 'Connected Customer (2006)
This article provides us with a better depth of understanding behind the theory of market research in the mobile application market. It hits upon several key concepts that will be important to incorporate and distinguish in our research.  This work also discusses the current trends and the future of mobile application marketing examining such topics as response fulfillment, research and data collection, store traffic generation, advertising, and branding, which will provide clues towards uncovering the psychological impulses that cause users to select one application over the other. It is a reliable first-hand resource from an organization that specializes in understanding what people want.