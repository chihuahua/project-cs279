\section{Related Work}

%related work syntheses focus on conceptual points
%highlight how different prior projects addressed similar issues (sometimes highlighting similarities, sometimes bringing together complementary evidence, sometimes pointing out lack of consistency across different publications)
%check out B. J. Fogg at Stanford, might have done some relevant research

Prior research has examined, to a certain extent, first impressions as they relate to the visual appeal of web pages. For instance, different cultures perceive the visual appeal of web pages in different ways \cite{Reinecke:2013:PUF:2470654.2481281}. Further, people from different groups see web pages' complexity and color saturations very differently even when they first view such a site \cite{Reinecke:2013:PUF:2470654.2481281}. Our research aims to delve deeper into first impressions and how it relates to people perceiving the apparent future success of a mobile application.

Developers already have insights as to the importance of gauging the success of their applications and how first impressions might be important \cite{wooldridge2010the}. For example, developers consider aspects such as the layout of their icons and delivering consistent messages to users. In addition, mobile application success has been discussed in marketing research in relation to the importance of icons, titles, descriptions, screen shots, keywords, and categories \cite{mureta12:app}.

Similarly, web designers know that first impressions matter and users make such assessments within a very short amount of time \cite{lindgaard}. Lindgaard further mentions how aesthetics are often neglected in current studies on emotion and design even though emotional responses can be triggered much more quickly than rational ones \cite{lindgaard}. Our study will capitalize on this insight as we examine how first impressions significantly impact people's emotional response to an application and underlying reasons for their belief that the application will be a success or not.