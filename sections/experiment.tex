\section{Experiment}

% This section is the meat of the paper
% This section should carefully describe the experiment and our analysis (design of the experiment)

Following this approach, we scraped 500 popular game apps that were recently released on the Android marketplace. We scraped the games' titles, icons, download figures, and their first screenshots. To gather our results, we built a site that asked subjects to select which app, in their opinion, had the highest likelihood of success through a continuous iterative process that ended when the user decided to end the experiment. We released the app to Facebook friends and classmates via Piazza. At first, we spread a version of the survey in which only the apps' names and icons were revealed to users. After that, we propogated a version that additionally revealed screenshots. We did so to see if incorporating a screenshot led to users having more accurate predictions of success.

For the analysis portion, we used a variant of the Bradley-Terry algorithm to assign ratings to the game apps. We established a uniformly distributed binary comparisons method automated with server-side logic. We hypothesized that since the views of a crowd strongly correlate with that of the downloaded figures, the crowd can provide a reasonable measure of the success of an app. We also compared the accuracy of subjects' predictions with the screenshot versus without it and confined some of our analysis to subjects who completed just ten continuous rounds.

To counter bias due to varying degrees of experience with smartphone applications, we asked subjects questions about their habits before beginning the study. We also only showed the title and icons for fifteen seconds in front of the user. Furthermore, our experiment consisted of an online study to be completed by volunteers through social media. We decided to gamify our study in order to better incentivize participants to complete the experiment. With the promise of revealing to our users their business savviness score, we convinced users to begin taking our experiment with the slogan: “How good is your entrepreneurial gut?"

For the crux of the study in which we have subjects compare apps, we asked subjects to choose the app they thought would be more successful.  Again, we wanted to ensure that participants who had knowledge of a particular app for a certain trial could not allow their knowledge of the apps previous standing to influence their decision. So, we discarded those trials, in which users self-reported knowledge of at least one of the apps. For each trial, there was a 20\% probability that our users would have to justify their selection. Their justification had to exceed five characters in order for them to continue. We only had subjects justify about 20\% of the time to elicit more meaningful explanations and deter participants from quitting the study.