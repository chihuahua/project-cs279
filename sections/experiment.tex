\section{Experiment}

% This section is the meat of the paper
% This section should carefully describe the experiment and our analysis (design of the experiment)

Following this approach, we scraped 500 popular game apps that were recently released on the Android marketplace. We scraped the games' titles, icons, download figures, and their first screenshots. We built a site that lets subjects compare two apps at a time based on likelihood of success over and over again. We released the app to Facebook friends as well as classmates on Piazza. First, we spread a version of the survey in which only apps' names and icons are revealed to users. After that, we spread a version of the survey that additionally reveals the first screenshot of the apps. We do so to see if incorporating a screenshot allows for people to better predict the success of apps. \\

At the end, we will use a variant of the Bradley-Terry algorithm to assign ratings to the game apps. Comparisons between apps will be chosen so as to distribute the selection of apps uniformly. We will see how well the views of the crowd correlate with the download figures, which we believe is a reasonable measure of the success of an app. We also plan to compare the accuracy of subjects' predictions with the screenshot versus without it. We will use the first ten rounds per user for doing so.  \\

To counter bias due to varying degrees of experience with smartphone apps, we ask subjects questions about their habits before beginning the study. We also only show the title and icons for five seconds in front of the user. Our experiment consists of an online study to be completed by volunteers through social media and potentially paid workers through Mechanical Turk. We decided to gamify our study in order to better incentivize participants to complete the experiment. With the promise to tell users “How good is your entrepreneurial gut?,” we convince users to begin taking our experiment.

Prior to the actual study, we conduct a brief survey on the participants' online habits in order to gather relevant data to assess whether these habits have any correlation with our findings. The few questions asked include how often participants use a smartphone, or download or use apps. Then, participants complete a brief demographic survey. Once again, only a few questions are asked. We mainly ask for their ages, genders, and countries of origin.

For the crux of the study in which we have subjects compare apps, we ask subjects the following questions for each comparison.

\begin{itemize}
\item Which app do you think is more successful?
\item Please briefly justify your choice.
\end{itemize}
  
Participants who have seen either app before will be presented with a new pair of apps. Otherwise, their previous experience could influence their selection. For each trial, there is a 20\% probability for which users must justify their selection. The justification must exceed five characters. We only have subjects justify about 20\% of the time to elicit more meaningful explanations and deter empty ones.