\section{Experiment}

% This section is the meat of the paper
% This section should carefully describe the experiment and our analysis (design of the experiment)

Following this approach, we scraped the titles, icons, download figures, and first screenshots of 500 popular game apps that were recently released in the Android marketplace.  The apps were all released between two weeks and a month prior to this study. To gather our results, we built a site that asked subjects to select which app, in their opinion, had the highest likelihood of success through an iterative process that ended when the user decided to quit the experiment.  We released the survey to our friends on Facebook and our classmates on Piazza. Although the initial release included only the apps' names and icons, the subsequent version we propagated displayed screenshots. The goal for the additional inclusion was to determine if incorporating screenshots could lead users to achieve more accurate predictions after our original results proved to be somewhat insignificant.  \\

135 individuals did the former survey in which only the names and icons were revealed. Out of these individuals, 50 were female, and 82 were male. Out of the 29 individuals who did the new survey, 17 were female and 12 were male. All individuals were solicited from social media channels such as Facebook, on the CS279 Piazza page, or from Amazon Mechanical Turk. The sources of individuals were randomly distributed per treatment. The vast majority of these users (about 83\%) use apps on their smart phones daily. \\

Uniformly distributed binary-comparison trials were automated with server-side logic. We hypothesized that people download apps using relatively quick, perfunctory measures; therefore, crowds can provide reasonably good measures of app success when shown an apps' icon, title, and screenshot. We also compared the accuracy of subjects' predictions with and without screenshots and confined parts of our analyses to subjects who completed at least ten contiguous rounds. Bias was a chief concern, so in order to counter it due to varying degrees of experience with smartphone applications, we asked subjects to identify if they had seen the app, for the current trial, before displaying the title and icon for fifteen seconds to the user. Because we were relying on attracting users solely through social media outlets, we decided to gamify our study to better incentivize potential participants. With the promise of revealing our user's business acumen, we drew users as well with our slogan: “How good is your entrepreneurial gut?"\\

Since we wanted to ensure that participants who had knowledge of a particular app for a certain trial could not allow that knowledge to influence their decision. For that reason, we discarded trials in which users had self-reported prior knowledge of at least one app. For each given trial, due to past unhelpful or blank responses, we made our users' responses exceed five characters. Although we originally had subjects justify their selection after every trial, we quickly switched our justifications to approximately 20\% of trials in hopes of eliciting more meaningful responses.\\
