\section{Experiment}

% This section is the meat of the paper
% This section should carefully describe the experiment and our analysis (design of the experiment)

Following this approach, we scraped 500 popular game apps that were recently released on the Android marketplace. We scraped the games' titles, icons, and download figures. We are building a site that lets subjects compare two apps at a time based on trustworthiness and perceived likelihood of success over and over again. We plan to release this survey to Turkers as well as a random distribution of people around us say in a dining hall. At the end, we will use a variant of Elo's algorithm to assign ratings to the game apps based on those two factors. Comparisons between apps will be chosen so as to distribute the selection of apps uniformly. We will see how well these factors correlate with the download figures, which we believe is a reasonable measure of the success of an app. \\

To counter bias, we will ask subjects the following questions about their habits before beginning the study. We will also only show the title and icons for five seconds in front of the user.

\begin{itemize}
\item How often do you use a smartphone?
\item How often do you download apps?
\item How often do you use apps?
\item How old are you?
\item What is your gender?
\item Where are you from?
\end{itemize}

For the actual study, we will ask subjects the following questions per comparison between two apps.

\begin{itemize}
\item Which app do you trust more?
\item Which app do you think is more successful?
\end{itemize}

Participants who have seen either app before will have their answers discarded. Otherwise, their previous experience could influence their selection. We plan to issue the trials in batches of five. We plan to encourage participants to do more batches by recording their scores (the percent they got right) and challenging them to test how strong their "entrepreneurial gut" is. We may even compare their entrepreneurial acuteness to other participants anonymously. Our goal is to recruit about 5000 participants.
