\documentclass{sigchi}

% Command to override the default ACM copyright statement (e.g. for preprints). 
% Consult the conference website for the camera-ready copyright statement.
\toappear{Permission to make digital or hard copies of all or part of this work for personal or classroom use is granted without fee provided that copies are not made or distributed for profit or commercial advantage and that copies bear this notice and the full citation on the first page. Copyrights for components of this work owned by others than ACM must be honored. Abstracting with credit is permitted. To copy otherwise, or republish, to post on servers or to redistribute to lists, requires prior specific permission and/or a fee. Request permissions from permissions@acm.org. \\
\emph{CHI'14}, April 26--May 1, 2014, Toronto, Canada. \\
Copyright \copyright~2014 ACM ISBN/14/04... \$15.00.}

\pagenumbering{arabic}

% Load basic packages
\usepackage{balance} % to better equalize the last page
\usepackage{graphics} % for EPS, load graphicx instead
\usepackage{times} % comment if you want LaTeX's default font
\usepackage{url} % llt: nicely formatted URLs

% llt: Define a global style for URLs, rather that the default one
\makeatletter
\def\url@leostyle{%
  \@ifundefined{selectfont}{\def\UrlFont{\sf}}{\def\UrlFont{\small\bf\ttfamily}}}
\makeatother
\urlstyle{leo}

% To make various LaTeX processors do the right thing with page size.
\def\pprw{8.5in}
\def\pprh{11in}
\special{papersize=\pprw,\pprh}
\setlength{\paperwidth}{\pprw}
\setlength{\paperheight}{\pprh}
\setlength{\pdfpagewidth}{\pprw}
\setlength{\pdfpageheight}{\pprh}

% Make sure hyperref comes last of your loaded packages, 
% to give it a fighting chance of not being over-written, 
% since its job is to redefine many LaTeX commands.
\usepackage[pdftex]{hyperref}
\hypersetup{
pdftitle={SIGCHI Conference Proceedings Format},
pdfauthor={LaTeX},
pdfkeywords={SIGCHI, proceedings, archival format},
bookmarksnumbered,
pdfstartview={FitH},
colorlinks,
citecolor=black,
filecolor=black,
linkcolor=black,
urlcolor=black,
breaklinks=true,
}

% create a shortcut to typeset table headings
\newcommand\tabhead[1]{\small\textbf{#1}}


% End of preamble. Here it comes the document.
\begin{document}

\title{Predicting the Success of Mobile Game Applications Based on First Impressions}

\numberofauthors{3}
\author{
  \alignauthor Paola Mariselli\\
    \affaddr{Harvard SEAS}\\
    \affaddr{35 Oxford St.}\\
    \email{paolamariselli@fas.harvard.edu}\\
  \alignauthor Sierra Okolo\\
    \affaddr{Harvard SEAS}\\
    \affaddr{35 Oxford St.}\\
    \email{sokolo@fas.harvard.edu}\\
  \alignauthor Chi Zeng\\
    \affaddr{Harvard College}\\
    \affaddr{Harvard Yard Pl.}\\
    \email{czeng@college.harvard.edu}\\
}

\maketitle
\begin{abstract}

The first impression that a game app makes plays an important role in its success, perhaps even more so than practical utility. Hence, the page of an app could determine much of the success of an app. Our project seeks to determine what specific factors about the app page matter in giving users a quality impression of a game app. We focus on Android games, but our results can be generalized to other platforms. \\

In fact, if users can predict the success of game apps based on features of app' pages such as names, icons, and screenshots, then developers can use the crowd to gauge the success of an app even before its release. We will see if users can effectively predict the success of apps based on just names and icons alone. We afterwards add in screenshots to see if they improve the accuracy of users' predictions. We seek to find if users' impressions of these specific features correlate with how often apps are downloaded. First impressions, we argue, really do matter.

\end{abstract}

\keywords{Mobile Applications; First Impression; Prediction}

\category{H.5.2.}{Information Interfaces and Presentation (e.g. HCI)}{User Interfaces}

\section{Introduction}

%what problem will your work address? why is it important?
%explain why first impressions matter: what first impressions lead users to do
%how people make their judgments (first impressions vs rational evaluation of a product)
%how those previous things make it reasonable to think our hypothesis
%what problem will your work address? why is it important?

First impressions often matter very much. From interview preparations to website aesthetics, people understand the power of an image. Moreover, people can form opinions within the first few seconds of viewing a particular product and use that impression to determine the potential for future company loyalties. That is the reason we decided to examine if first impressions mattered in the context of mobile game applications. In fact, knowledge of this concept has led independent mobile application developers to attempt to solicit feedback on specific features of their application through advertisements and crowdsourcing services. Nevertheless, such services can be costly and the process, being unformalized, does not offer the developer a holistic view of the potential success of their application. It simply offers them an idea of users' opinions regarding a feature in isolation. Therefore, the ability to forecast the level of success that a future mobile application can achieve ahead of deployment can be an extremely valuable tool \cite{tohidi06:getting}. Although, the method of predicting success can vary, we wanted to determine if crowds can be leveraged to predict app success with trivial app descriptors.

The typical existing approach involves deploying an application and potentially failing. To avoid the time-consuming and expensive process of deploying an application that might fail, mobile application developers, for example, take many measures to gauge the success of their applications prior to release. One of these methods includes the release of a beta version of the application to a restricted community before releasing the app to the public as this assists them in developing a more accurate estimation of future interest \cite{betatest}. However, conducting a meaningful beta test requires a reliable and closed community of users. Although many large firms have such resources, independent mobile application developers are reduced to deploying their application and repeatedly modifying it in hopes that, through trial-and-error, the application might become more successful \cite{betatest}.

%what's your insight? how are you going to go after the problem?
For our approach, we will use Android mobile games to test our experiment. We choose to use mobile games for our experiment as the data is widely available. As users' first impressions of mobile applications are most often made through the application store, our study will simulate the features found on such pages. We seek to establish this link by conducting a series of online studies. Success will be measured by an app's popularity, represented by the number downloads it obtains. Measurements, which we believe crowds can use to accurately predict success, include the title, icon, and screenshot of the app. Our studies will also include a demographics survey in order to better assess the mechanisms underlying the connection. For instance, women and men may have systematic preferences for certain designs over others that allow them to more accurately predict the success of an app.

%lasting knowledge that will result from work
We make the following contributions:
\begin{itemize}
\item We attempt to establish a link between the way users feel about a given application measurement and the application's actual success.
\item We attempt to deliver a novel approach to predicting the success of mobile applications.
\end{itemize}

\section{Related Work}

%related work syntheses focus on conceptual points
%highlight how different prior projects addressed similar issues (sometimes highlighting similarities, sometimes bringing together complementary evidence, sometimes pointing out lack of consistency across different publications)
%check out B. J. Fogg at Stanford, might have done some relevant research

%% == SIERRA: who is this paper by? ==%%
%% == SIERRA: can you make sure each of these related works have its properly formatted entry on our .bib document? ==%%
Quantifying Visual Preferences around the World
This paper analyzes how people from different demographics react to how colorful web pages are. The experiment collected ratings of visual appeal from about forty thousand subjects on 430 web pages of varying complexities and colorfulness. The qualities that subjects performed ratings on include trustworthiness and  of the web page. The experiment found that certain demographics such as Russians and Fins did not find colorful web pages as appealing as did demographics such as Macedonians.
This paper relates to our work since we too seek to find factors that heighten appeal. Specifically, we seek factors that maximize the appeal of pages for game and productivity apps on the Android platform. Icon color could be one of the factors we analyze. We could also borrow off of many ideas from the methodology. For instance, the researchers motivated individuals to do their test by comparing their results to other individuals at large. Our method for performing regression might be more complicated though since we analyze more factors than just one (say color saturation of a page).
Overall, I enjoyed reading this paper. The results showed that most people preferred a moderate level of color saturation in web pages. I feel that some of the conclusions about specific demographics of people and their color preferences succumbed to a small amount of response bias though. In some countries, only a small portion of residents have access to the internet.

The Business of iPhone App Development: Making and Marketing Apps that Succeed
By Dave Woolridge and Michael Schneider (2010)
This book offered us some keen insights on how app developers currently gauge the success of their apps both after and before release. For instance, many services out there such as Mobclix provide rankings of apps based on download figures. App developers also read over the reviews of competitors to determine how well their services will be received.
App developers also emphasize first impression a lot according to this book. For instance, they value the layout of their icons. Developers should also value communicating a consistent message to the user. From this book, we determined several factors that could heavily influence first impressions of app pages. I wished that the book discussed how crowd sourcing could help app developers gauge app success though. This book was written in 2010, and gaining knowledge from crowds was not as substantial of a concept back then.

Predicting Users's First Impressions of Website Aesthetics With a Quantification of Perceived Visual Complexity and Colorfulness
This paper examined how different groups of people perceived web pages of varying complexity and color saturations. The experimenters asked 548 participants to rate 450 different websites on a number of metrics. What intrigued me most about this paper was how it used quantitative measures to ascertain such soft qualities as page complexity and colorfulness. Perceived colorfulness even depended on the context of the colors. The experimenters nicely used the sum of the average and the standard deviation of the saturations to measure perceived colorfulness.
In our study, we will have to take similar measures. We are also trying to quantitatively gauge the effects of such soft qualities as trustworthiness and fun-ness of an app's page.

Attention web designers: You have 50 milliseconds to make a good first impression
This paper examined how viewers of web pages really make their first impressions about a website within a very short amount of time. The paper emphasizes how aesthetics is often neglected in current studies on emotion and design. Apparently, emotional responses can be triggered much more quickly than rational ones. Humans are quick to assign words such as clean, symmetric, and dark to images. This article hence directly relates to our current studies since it discusses how first impressions can significantly impact people's emotional response to an app. Hence, we should ask users to rate their emotional responses to various apps and/or their icons. However, I feel that this paper also somewhat understates the importance of functionality. I wish it could further examine how important this emotional response is.~\cite{tohidi06:getting}

App Empire: Make Money, Have a Life, and Let Technology Work for You
By Chad Mureta (2012)
Since marketing plays such an essential role in the research we intend to pursue, we thought it best to examine literature related to mobile applications marketing research. In this book, Mureta highlights key insights regarding mobile application design and marketing which separate successful apps from non-successful ones.  Using raw, forthright diction, Mureta acts as sort of a personal mentor to the reader, using words that convey a sort of familial bond that is greatly disarming, psychologically enticing, but more importantly packed with valuable tips for fellow application entrepreneurs.  It is this direct, candid advice that we seek to capture.  Furthermore, with his pointed market overviews that focus on mobile application usage and financial statistics and his instructional section on "Sex App-eal" in which he maintains a discourse on the importance of icons, titles, descriptions, screen shots, keywords,and categories, Mureta's work will endow us with the knowledge and references we need to augment our understanding of mobile application design and develop better, more informed hypotheses regarding what makes a mobile application successful. Finally, the reliability of Mureta's claims are backed by his own success in the mobile application industry.~\cite{mureta12:app}

Mobile Marketing Research Priorities: Roadmap to Engaging the 'Connected Customer (2006)
This article provides us with a better depth of understanding behind the theory of market research in the mobile application market. It hits upon several key concepts that will be important to incorporate and distinguish in our research.  This work also discusses the current trends and the future of mobile application marketing examining such topics as response fulfillment, research and data collection, store traffic generation, advertising, and branding, which will provide clues towards uncovering the psychological impulses that cause users to select one application over the other. It is a reliable first-hand resource from an organization that specializes in understanding what people want.
\section{Approach}

In order to tackle this problem we will perform the following: We intend to scrape 500 mobile game applications from the Android market. We are aiming for a total of 5000 participants for our study. For each trial, a subject will be asked to compare two factors within two applications.  These factors are the following: (1) which application will be more successful and (2) why. The user will be only shown the titles and icons of the apps for five\?\? seconds. During the trials, users will be asked to briefly explain their selections. We plan to issue the trials in batches of five, but we will ask users whether they have seen the application before and discard that sample if they say they have. We hope that the client-side randomization applications shown will reduce the chance of users seeing the same application multiple times. The survey will be powered by Django and a MySQL database.\\

We will make our trials more appealing by treating each trial as a test of how good users' entrepreneurial guts are. After users answer which application they think is more successful, our system will show them the real answer and ask them the reason they chose a certain option. We hope that the gamification of this experiment will lead to more users willing to participate in our survey and more experimental data to choose from. Subjects will also see a score denoting what proportion of trials they guessed correctly. We will do this through an iterative process, informing the user at each step what their score is to create a sort of competitive environment.\\

Afterwards, we will use an algorithm to rank the applications based on which ones users found to be more successful. We will see how well these rankings correlate with how successful the apps actually are in terms of download counts.

%\begin{itemize}
%HOW GOOD IS YOUR ENTREPRENEURIAL GUT?
%agreement to statement of privacy
%two name of apps + their icons
%option to opt-out if they've already seen app
%which one do you think is more successful? (select one)
%why do you think this one is more successful? (free-form comment)
%result of whether they were right or wrong + score + try again!

%\item Control for difference in application cost by only using free applications
%\item Obtain approximately 500 mobile applications and 5000 participants 
%\item Utilize the application store: Preference towards Androids since they publicly provide download statistics
%\item Use samples from games and productivity applications
%\item Incentivize users by making the study fun to complete and additionally taking advantage of users from Mechanical Turk to cover the remaining numbers required.
%\item Performing a quick survey for the first five users
%\item Display applications in different orders to weed out systematic preferences based on order.
%\item Utilizing apps with roughly the same distribution of dates of release
%\item Separating applications into two categories: (1) relatively new to the market, so users have not seen them before, and (2) been in the market for a while
%\item Displaying a mock-up of the application store listing (all must be the same except for the variable being tested)
%\item Scaping the the names, icons, and descriptions of the aforementioned applications from the Android store
%\end{itemize}
%
%We plan to conduct a survey across crowds that gauges their views on various aspects of web pages of applications. We control for differences in application cost by only examining free applications.
%We will scrape about 500 applications and get about 5000 participants to do the survey (Our sample must be big to offset random error.).
%We plan to use Android apps since its application marketplace publicly provides download statistics, albeit in discretized increments.
%We plan to examine a specific subset of apps: games + productivity app (The use of games is discretionary, so we must also examine something else that people need to use). \\
%
%We will incentivize users by making the study fun to complete. If not enough users complete the study in this way, we will proceed to obtain some users via Mechanical Turk.
%setup: quick survey + first batch of 5
%We plan to show applications in different orders to different people (to eliminate systematic biases).
%We will try to have roughly the same distribution of dates for when the applications were released.
%We plan to separate applications into 2 categories: (1) relatively new to the market, so users have not seen them before, and (2) been in the market for a while
%show mock-up of application store listing (all must be the same except for the variable being tested) if Android releases information on older applications.
%To obtain data, we plan to scrape the names, icons, and descriptions of the aforementioned applications.

\section{Experiment}

% This section is the meat of the paper
% This section should carefully describe the experiment and our analysis (design of the experiment)
\section{Results and Analysis}

%The data should be rich enough so that you can reflect on which aspects of your project are working and which arent
%This should be a brief summary of the results and what they mean for your continuing progress

We found that screenshots did not significantly increase the accuracy of subjects' predictions of apps' relative successes. On average, the 135 individuals exposed to apps' first screenshots predicted 46.1\% of the rounds correctly. For the 29 individuals exposed to just the names and icons of apps, the figure was 42.6\%. The p-value for two-tailed t-test assuming equal variances was 0.324, so the difference between the two groups was not significant. The 50 females and 82 males who were only exposed to apps' names and icons also did not exhibit significant differences between each other in their abilities to predict app success. \\

Unfortunately, since the average overall accuracy is under 50\%, we cannot build a system for gauging the success of apps based on names, icons, and screenshots alone. It seems that factors beyond these features of an app matter. Perhaps the quality of the app or marketing strategies far trump the features on an app page in terms of determining how many downloads apps receive. Further research can build on these results by finding out factors that do matter for game apps. Perhaps a study that examines the success of apps just a few hours after their releases would more effectively determine the contributions of icons and screenshots towards the success of apps. In those cases, far fewer users would have downloaded the apps, so the effect of the first impression that app pages offer would be more pronounced. This study is hard to conduct, however, since most market places for apps do not reveal the specific times of release of apps.

We also performed an ordinal logistic regression on how often individuals downloaded and used apps on their phones. Very interestingly, we discovered that individuals that downloaded and used apps more often did not have significantly higher accuracies. From this result, we conclude that individuals of the general populace are not experts at predicting which apps will be downloaded more often than others. Those individuals who use smartphone apps daily are no exception. Any tools that try to gauge the success of apps based on these features of the app page alone will likely be ineffective. \\


\bibliographystyle{plain}
\bibliography{finalproject}

\end{document}


%% == Chi's section of examples == %%
% = Includin a Figure = %
%\begin{figure}
%\centering
%\includegraphics[width=1in]{space.jpg} 

%\caption{Another sample figure}
%\label{figure-sample2}
%\end{figure}

% = Generate a bibliography. = %
%\bibliography{proposal}
%\bibliographystyle{unsrt}

%% == Paola's section of examples == %%
%\textcolor{red}{Mandatory section to be included in your final version.}
%~\cite{tohidi06:getting,tohidi06:user}
