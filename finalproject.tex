\documentclass{sigchi}

% Command to override the default ACM copyright statement (e.g. for preprints). 
% Consult the conference website for the camera-ready copyright statement.
\toappear{Permission to make digital or hard copies of all or part of this work for personal or classroom use is 	granted without fee provided that copies are not made or distributed for profit or commercial advantage and that copies bear this notice and the full citation on the first page. Copyrights for components of this work owned by others than ACM must be honored. Abstracting with credit is permitted. To copy otherwise, or republish, to post on servers or to redistribute to lists, requires prior specific permission and/or a fee. Request permissions from permissions@acm.org. \\
{\emph{CHI'14}}, April 26--May 1, 2014, Toronto, Canada. \\
Copyright \copyright~2014 ACM ISBN/14/04...\$15.00.
}

% Arabic page numbers for submission. 
% Remove this line to eliminate page numbers for the camera ready copy
\pagenumbering{arabic}

% Load basic packages
\usepackage{balance}  % to better equalize the last page
\usepackage{graphics} % for EPS, load graphicx instead
\usepackage{times}    % comment if you want LaTeX's default font
\usepackage{url}      % llt: nicely formatted URLs

% llt: Define a global style for URLs, rather that the default one
\makeatletter
\def\url@leostyle{%
  \@ifundefined{selectfont}{\def\UrlFont{\sf}}{\def\UrlFont{\small\bf\ttfamily}}}
\makeatother
\urlstyle{leo}

% To make various LaTeX processors do the right thing with page size.
\def\pprw{8.5in}
\def\pprh{11in}
\special{papersize=\pprw,\pprh}
\setlength{\paperwidth}{\pprw}
\setlength{\paperheight}{\pprh}
\setlength{\pdfpagewidth}{\pprw}
\setlength{\pdfpageheight}{\pprh}

% Make sure hyperref comes last of your loaded packages, 
% to give it a fighting chance of not being over-written, 
% since its job is to redefine many LaTeX commands.
\usepackage[pdftex]{hyperref}
\hypersetup{
pdftitle={SIGCHI Conference Proceedings Format},
pdfauthor={LaTeX},
pdfkeywords={SIGCHI, proceedings, archival format},
bookmarksnumbered,
pdfstartview={FitH},
colorlinks,
citecolor=black,
filecolor=black,
linkcolor=black,
urlcolor=black,
breaklinks=true,
}

% create a shortcut to typeset table headings
\newcommand\tabhead[1]{\small\textbf{#1}}


% End of preamble. Here it comes the document.
\begin{document}

\title{Predicting Mobile Application Success Based on First Impressions}

\numberofauthors{3}
\author{
  \alignauthor Paola Mariselli\\
    \affaddr{Harvard SEAS}\\
    \affaddr{35 Oxford St.}\\
    \email{paolamariselli@fas.harvard.edu}\\
  \alignauthor Sierra Okolo\\
    \affaddr{Harvard SEAS}\\
    \affaddr{35 Oxford St.}\\
    \email{sokolo@fas.harvard.edu}\\
  \alignauthor Chi Zeng\\
    \affaddr{Harvard College}\\
    \affaddr{Harvard Yard Pl.}\\
    \email{czeng@college.harvard.edu}\\
}

\maketitle

\begin{abstract}
This abstract is in construction.
\end{abstract}

\keywords{Mobile Application; First Impression; Prediction}

\category{H.5.2.}{Information Interfaces and Presentation (e.g. HCI)}{User Interfaces}

\section{Introduction}

%% == SIERRA: please figure out how to properly format bullet points for contributions below per SIGCHI's formatting ==%%
%% == SIERRA: please add sources where you see cite{need source}. some sources may be in the related work, some may need to be added ==%%

%what problem will your work address? why is it important?
%explain why first impressions matter: what first impressions lead users to do
%how people make their judgments (first impressions vs rational evaluation of a product)
%how those previous things make it reasonable to think our hypothesis
%what problem will your work address? why is it important?
%what's your insight? how are you going to go after the problem?
%lasting knowledge that will result from work

It has long been a dream of application creators to know whether a given product will be successful or not ahead of its deployment. Knowing whether a given website or application will be a success prior to releasing a product out into the wild could potentially save designers billions of dollars~\cite{needsource}. But the question still lies unanswered: how do you predict the success of an application without first deploying it?

The existing approach involves deploying an application and potentially failing. To avoid the time-consuming and expensive process of deploying an application that might fail, mobile application developers, for example, take many measures to gauge the success of their applications prior to release, one of which is releasing a beta version of their application to a restricted community before releasing the app to the public in order to estimate future interest~\cite{needsource}. However, conducting a meaningful beta test requires a reliable and closed community of users. Although many large firms have such resources~\cite{needsource}, independent mobile application developers have to just deploy their application and modify it, if possible, so that through trial-and-error the application might become more successful.

Further, it has been well-established that first impressions matter. From interviews to website aesthetics, users form an opinion within the first few seconds of viewing a given stimuli~\cite{needsource}. Thus far, research on first impressions has only been examined within the context of websites~\cite{needsource}, we seek to further first impression insights by exploring them within the context of mobile applications.

Since designers know that first impressions matter, independent mobile application developers have tried obtaining feedback on specific features of their application through advertisements and crowdsourcing services~\cite{needsource}. Nevertheless, such services can be costly, and their feedback process is not formalized or offers the developer a holistic view of the potential success of their application, simply of the feature in isolation.

Our goal is to combine the insights on product success and first impressions by predicting the success of mobile applications based on users' first impressions. Being able to predict the success of mobile applications  based on observable measures will allow us to establish a link between the first impressions of objective measures and the application’s eventual success. In addition, our approach has the potential to improve the existing design process by helping designers get the right design prior to deployment.

For our approach, we will use Android mobile games to test during our experiment. We chose to use mobile games for our experiment as the data is widely available. As users' first impressions of mobile applications are most often made through the application store, our study will simulate the layout found on such pages to replicate conditions and streamline variables. We seek to conduct a series of online studies, where the success of an application is measured by its popularity, represented by the number of its downloads. Potential measurements to predict success used include icon, slogan, screenshots (aesthetic appeal), and description. Our studies will also include a demographics survey in order to better assess the mechanisms underlying the connection. For instance, people may think a given application will be successful because they perceive it as more trustworthy or fun.

We make the following contributions: \\ \\
- We establish a link between the way users feel about a given application measurement and the application's actual success.\\
- We deliver a novel approach to predicting the success of mobile applications.

\section{Related Work}

%related work syntheses focus on conceptual points
%highlight how different prior projects addressed similar issues (sometimes highlighting similarities, sometimes bringing together complementary evidence, sometimes pointing out lack of consistency across different publications)
%check out B. J. Fogg at Stanford, might have done some relevant research

Prior research has examined, to a certain extent, first impressions as it relates to the visual appeal of web pages. For instance, different cultures perceive the visual appeal of web pages in different ways \cite{Reinecke:2013:PUF:2470654.2481281}. Further, people from different groups see web pages' complexity and color saturations very differently even during the very first impression of such sites \cite{Reinecke:2013:PUF:2470654.2481281}. Our research aims to delve deeper into first impressions and how it relates to people perceiving the apparent future success of a mobile application.

Developers already have insights as to the importance of gauging success of their application and how first impressions might be important \cite{wooldridge2010the}. For example, application developers consider aspects such as the layout of their icons or how to deliver a consistent messages to users when developing an application. However, such insights have not been further developed. In addition, mobile application success has been discussed in marketing research in relation to the importance of icons, titles, descriptions, screen shots, keywords, and categories \cite{mureta12:app}.

Similarly, web designers know that first impressions matter and users make such assessments within a very short amount of time \cite{needsource-attentionwebdesigners}. Lindgaard \cite{needsource-attentionwebdesigners} further mentions how aesthetics are often neglected in current studies on emotion and design even though emotional responses can be triggered much more quickly than rational ones. Our study will capitalize on this insight as we will examine how first impressions  significantly impact people's emotional response to an application as underlying reasons for their belief that the application will be a success or not.


\bibliographystyle{acm-sigchi}
\bibliography{finalproject}

\end{document}


%% == Chi's section of examples == %%
% = Includin a Figure = %
%\begin{figure}
%\centering
%\includegraphics[width=1in]{space.jpg} 
%
%\caption{Another sample figure}
%\label{figure-sample2}
%\end{figure}

% = Generate a bibliography. = %
%\bibliography{proposal}
%\bibliographystyle{unsrt}

%% == Paola's section of examples == %%
%\textcolor{red}{Mandatory section to be included in your final version.}