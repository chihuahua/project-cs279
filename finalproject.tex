\documentclass{sigchi}

% Command to override the default ACM copyright statement (e.g. for preprints). 
% Consult the conference website for the camera-ready copyright statement.
\toappear{Permission to make digital or hard copies of all or part of this work for personal or classroom use is granted without fee provided that copies are not made or distributed for profit or commercial advantage and that copies bear this notice and the full citation on the first page. Copyrights for components of this work owned by others than ACM must be honored. Abstracting with credit is permitted. To copy otherwise, or republish, to post on servers or to redistribute to lists, requires prior specific permission and/or a fee. Request permissions from permissions@acm.org. \\
\emph{CHI'14}, April 26--May 1, 2014, Toronto, Canada. \\
Copyright \copyright~2014 ACM ISBN/14/04... \$15.00.}

\pagenumbering{arabic}

% Load basic packages
\usepackage{balance} % to better equalize the last page
\usepackage{graphics} % for EPS, load graphicx instead
\usepackage{times} % comment if you want LaTeX's default font
\usepackage{url} % llt: nicely formatted URLs

% llt: Define a global style for URLs, rather that the default one
\makeatletter
\def\url@leostyle{%
  \@ifundefined{selectfont}{\def\UrlFont{\sf}}{\def\UrlFont{\small\bf\ttfamily}}}
\makeatother
\urlstyle{leo}

% To make various LaTeX processors do the right thing with page size.
\def\pprw{8.5in}
\def\pprh{11in}
\special{papersize=\pprw,\pprh}
\setlength{\paperwidth}{\pprw}
\setlength{\paperheight}{\pprh}
\setlength{\pdfpagewidth}{\pprw}
\setlength{\pdfpageheight}{\pprh}

% Make sure hyperref comes last of your loaded packages, 
% to give it a fighting chance of not being over-written, 
% since its job is to redefine many LaTeX commands.
\usepackage[pdftex]{hyperref}
\hypersetup{
pdftitle={SIGCHI Conference Proceedings Format},
pdfauthor={LaTeX},
pdfkeywords={SIGCHI, proceedings, archival format},
bookmarksnumbered,
pdfstartview={FitH},
colorlinks,
citecolor=black,
filecolor=black,
linkcolor=black,
urlcolor=black,
breaklinks=true,
}

% create a shortcut to typeset table headings
\newcommand\tabhead[1]{\small\textbf{#1}}


% End of preamble. Here it comes the document.
\begin{document}

\title{Predicting Mobile Application Success Based on First Impressions}

\numberofauthors{3}
\author{
  \alignauthor Paola Mariselli\\
    \affaddr{Harvard SEAS}\\
    \affaddr{35 Oxford St.}\\
    \email{paolamariselli@fas.harvard.edu}\\
  \alignauthor Sierra Okolo\\
    \affaddr{Harvard SEAS}\\
    \affaddr{35 Oxford St.}\\
    \email{sokolo@fas.harvard.edu}\\
  \alignauthor Chi Zeng\\
    \affaddr{Harvard College}\\
    \affaddr{Harvard Yard Pl.}\\
    \email{czeng@college.harvard.edu}\\
}

\maketitle
\begin{abstract}
The factors that govern the success of a mobile application undergo a highly meritocratic and democratized process. It is well known that the public volitionally invests in applications with superior designs and utility regardless of the amount of money was invested in the product. Or do they? Particularly when considering game applications, why does a given user choose one application over another when only the very first impression, icon, and title are considered? \\

We initiated this project in an attempt to gain clarity on a wide-scale macroeconomic event that occurs everyday: downloading a game mobile application. By examining observable measures on the application page, which allow users to make a very first impression, we can determine if a correlation exists between icons, titles, and the future success of that application. This research has the potential to save application designers and developers vast resources, especially those expended towards market research that attempts to understand what incentivizes users to connect to applications. First impressions, we argue, really do matter.
\end{abstract}

\keywords{Mobile Applications; First Impression; Prediction}

\category{H.5.2.}{Information Interfaces and Presentation (e.g. HCI)}{User Interfaces}

\section{Introduction}

%% == SIERRA: please figure out how to properly format bullet points for contributions below per SIGCHI's formatting ==%%
%% == SIERRA: please add sources where you see cite{need source}. some sources may be in the related work, some may need to be added ==%%

%what problem will your work address? why is it important?
%explain why first impressions matter: what first impressions lead users to do
%how people make their judgments (first impressions vs rational evaluation of a product)
%how those previous things make it reasonable to think our hypothesis
%what problem will your work address? why is it important?
%what's your insight? how are you going to go after the problem?
%lasting knowledge that will result from work

It has long been a dream of application creators to know whether a given product will be successful or not ahead of its deployment. Knowing whether a given website or application will be a success prior to releasing a product out into the wild could potentially save designers billions of dollars~\cite{needsource}. But the question still lies unanswered: how do you predict the success of an application without first deploying it?

The existing approach involves deploying an application and potentially failing. To avoid the time-consuming and expensive process of deploying an application that might fail, mobile application developers, for example, take many measures to gauge the success of their applications prior to release, one of which is releasing a beta version of their application to a restricted community before releasing the app to the public in order to estimate future interest~\cite{needsource}. However, conducting a meaningful beta test requires a reliable and closed community of users. Although many large firms have such resources~\cite{needsource}, independent mobile application developers have to just deploy their application and modify it, if possible, so that through trial-and-error the application might become more successful.

Further, it has been well-established that first impressions matter. From interviews to website aesthetics, users form an opinion within the first few seconds of viewing a given stimuli~\cite{needsource}. Thus far, research on first impressions has only been examined within the context of websites~\cite{needsource}, we seek to further first impression insights by exploring them within the context of mobile applications.

Since designers know that first impressions matter, independent mobile application developers have tried obtaining feedback on specific features of their application through advertisements and crowdsourcing services~\cite{needsource}. Nevertheless, such services can be costly, and their feedback process is not formalized or offers the developer a holistic view of the potential success of their application, simply of the feature in isolation.

Our goal is to combine the insights on product success and first impressions by predicting the success of mobile applications based on users' first impressions. Being able to predict the success of mobile applications  based on observable measures will allow us to establish a link between the first impressions of objective measures and the application’s eventual success. In addition, our approach has the potential to improve the existing design process by helping designers get the right design prior to deployment.

For our approach, we will use Android mobile games to test during our experiment. We chose to use mobile games for our experiment as the data is widely available. As users' first impressions of mobile applications are most often made through the application store, our study will simulate the layout found on such pages to replicate conditions and streamline variables. We seek to conduct a series of online studies, where the success of an application is measured by its popularity, represented by the number of its downloads. Potential measurements to predict success used include icon, slogan, screenshots (aesthetic appeal), and description. Our studies will also include a demographics survey in order to better assess the mechanisms underlying the connection. For instance, people may think a given application will be successful because they perceive it as more trustworthy or fun.

We make the following contributions: \\ \\
- We establish a link between the way users feel about a given application measurement and the application's actual success.\\
- We deliver a novel approach to predicting the success of mobile applications.

\section{Related Work}

%related work syntheses focus on conceptual points
%highlight how different prior projects addressed similar issues (sometimes highlighting similarities, sometimes bringing together complementary evidence, sometimes pointing out lack of consistency across different publications)
%check out B. J. Fogg at Stanford, might have done some relevant research

Prior research has examined, to a certain extent, first impressions as it relates to the visual appeal of web pages. For instance, different cultures perceive the visual appeal of web pages in different ways \cite{Reinecke:2013:PUF:2470654.2481281}. Further, people from different groups see web pages' complexity and color saturations very differently even during the very first impression of such sites \cite{Reinecke:2013:PUF:2470654.2481281}. Our research aims to delve deeper into first impressions and how it relates to people perceiving the apparent future success of a mobile application.

Developers already have insights as to the importance of gauging success of their application and how first impressions might be important \cite{wooldridge2010the}. For example, application developers consider aspects such as the layout of their icons or how to deliver a consistent messages to users when developing an application. However, such insights have not been further developed. In addition, mobile application success has been discussed in marketing research in relation to the importance of icons, titles, descriptions, screen shots, keywords, and categories \cite{mureta12:app}.

Similarly, web designers know that first impressions matter and users make such assessments within a very short amount of time \cite{needsource-attentionwebdesigners}. Lindgaard \cite{needsource-attentionwebdesigners} further mentions how aesthetics are often neglected in current studies on emotion and design even though emotional responses can be triggered much more quickly than rational ones. Our study will capitalize on this insight as we will examine how first impressions  significantly impact people's emotional response to an application as underlying reasons for their belief that the application will be a success or not.

\section{Approach}

In order to tackle this problem we first scraped 500 mobile game applications from the Android market. For each trial, a subject was asked to compare two factors between two distinct applications. These factors are the following: (1) which application will be more successful and (2) why. The user was only shown the titles and icons of the apps for fifteen seconds. During the trials, users were asked to briefly explain their selections 20\% of the time.Client-side randomization was used for applications shown to reduce the chance of users seeing the same application multiple times. The survey was powered by Django and a MySQL database.

We made our trials more appealing by treating each trial as a test of how good users were at business savvy. After users answered which application they thought is more successful, our system show them the real answer and asked them, on occassion, the reason they chose a certain option. After 10 trials, subjects received a score denoting what proportion of trials they guessed correctly. We did this through an iterative process, informing the user at each step after the tenth trial what their score was to create a more addicting study.

%\begin{itemize}
%HOW GOOD IS YOUR ENTREPRENEURIAL GUT?
%agreement to statement of privacy
%two name of apps + their icons
%option to opt-out if they've already seen app
%which one do you think is more successful? (select one)
%why do you think this one is more successful? (free-form comment)
%result of whether they were right or wrong + score + try again!

%\item Control for difference in application cost by only using free applications
%\item Obtain approximately 500 mobile applications and 5000 participants 
%\item Utilize the application store: Preference towards Androids since they publicly provide download statistics
%\item Use samples from games and productivity applications
%\item Incentivize users by making the study fun to complete and additionally taking advantage of users from Mechanical Turk to cover the remaining numbers required.
%\item Performing a quick survey for the first five users
%\item Display applications in different orders to weed out systematic preferences based on order.
%\item Utilizing apps with roughly the same distribution of dates of release
%\item Separating applications into two categories: (1) relatively new to the market, so users have not seen them before, and (2) been in the market for a while
%\item Displaying a mock-up of the application store listing (all must be the same except for the variable being tested)
%\item Scaping the the names, icons, and descriptions of the aforementioned applications from the Android store
%\end{itemize}
%
%We plan to conduct a survey across crowds that gauges their views on various aspects of web pages of applications. We control for differences in application cost by only examining free applications.
%We will scrape about 500 applications and get about 5000 participants to do the survey (Our sample must be big to offset random error.).
%We plan to use Android apps since its application marketplace publicly provides download statistics, albeit in discretized increments.
%We plan to examine a specific subset of apps: games + productivity app (The use of games is discretionary, so we must also examine something else that people need to use). \\
%
%We will incentivize users by making the study fun to complete. If not enough users complete the study in this way, we will proceed to obtain some users via Mechanical Turk.
%setup: quick survey + first batch of 5
%We plan to show applications in different orders to different people (to eliminate systematic biases).
%We will try to have roughly the same distribution of dates for when the applications were released.
%We plan to separate applications into 2 categories: (1) relatively new to the market, so users have not seen them before, and (2) been in the market for a while
%show mock-up of application store listing (all must be the same except for the variable being tested) if Android releases information on older applications.
%To obtain data, we plan to scrape the names, icons, and descriptions of the aforementioned applications.

\section{Experiment}

% This section is the meat of the paper
% This section should carefully describe the experiment and our analysis (design of the experiment)

Following this approach, we scraped 500 popular game apps that were recently released on the Android marketplace. We scraped the games' titles, icons, and download figures. We are building a site that lets subjects compare two apps at a time based on likelihood of success over and over again. We plan to release this survey to Turkers as well as a random distribution of people around us say in a dining hall. At the end, we will use a variant of Elo's algorithm to assign ratings to the game apps. Comparisons between apps will be chosen so as to distribute the selection of apps uniformly. We will see how well the views of the crowd correlate with the download figures, which we believe is a reasonable measure of the success of an app. \\

To counter bias due to varying degrees of experience with smartphone apps, we will ask subjects questions about their habits before beginning the study. We will also only show the title and icons for five seconds in front of the user. Our experiment will consist of an online study to be completed by volunteers through social media and potentially paid workers through Mechanical Turk. We decided to gamify our study in order to better incentivize, particularly volunteer, participants to complete the experiment. With the promise to tell users “How good is your entrepreneurial gut?,” we convince users to begin taking our experiment.

Prior to the actual study, we conduct a brief survey on the participants' online habits in order to gather relevant data to assess whether these habits have any correlation with our findings. The few questions asked include how often participants use a smartphone, or download or use apps. Then, participants complete a brief demographic survey. Once again, only a few questions are asked -- mainly, their age, gender, and country of origin.

For the actual study, we will ask subjects the following questions per comparison between two apps.

\begin{itemize}
\item Which app do you think is more successful?
\item Please briefly justify your choice.
\end{itemize}

Participants who have seen either app before will be presented with a new pair of apps. Otherwise, their previous experience could influence their selection. We plan to issue the trials in batches of five. We plan to encourage participants to do more batches by recording their scores (the percent they got right) and challenging them to test how strong their "entrepreneurial gut" is. We may even compare their entrepreneurial acuteness to other participants anonymously. Our goal is to recruit about 5000 participants.

\section{Results and Analysis}

%The data should be rich enough so that you can reflect on which aspects of your project are working and which arent
%This should be a brief summary of the results and what they mean for your continuing progress

We found that screenshots did not significantly increase the accuracy of subjects' predictions of apps' relative successes. On average, the 135 individuals exposed to apps' first screenshots predicted 46.1\% of the rounds correctly. For the 29 individuals exposed to just the names and icons of apps, the figure was 42.6\%. The p-value for two-tailed t-test assuming equal variances was 0.324, so the difference between the two groups was not significant. The 50 females and 82 males who were only exposed to apps' names and icons also did not exhibit significant differences between each other in their abilities to predict app success. \\

Unfortunately, since the average overall accuracy is under 50\%, we cannot build a system for gauging the success of apps based on names, icons, and screenshots alone. It seems that factors beyond these features of an app matter. Perhaps the quality of the app or marketing strategies far trump the features on an app page in terms of determining how many downloads apps receive. Further research can build on these results by finding out factors that do matter for game apps. Perhaps a study that examines the success of apps just a few hours after their releases would more effectively determine the contributions of icons and screenshots towards the success of apps. In those cases, far fewer users would have downloaded the apps, so the effect of the first impression that app pages offer would be more pronounced. This study is hard to conduct, however, since most market places for apps do not reveal the specific times of release of apps.

We also performed an ordinal logistic regression on how often individuals downloaded and used apps on their phones. Very interestingly, we discovered that individuals that downloaded and used apps more often did not have significantly higher accuracies. From this result, we conclude that individuals of the general populace are not experts at predicting which apps will be downloaded more often than others. Those individuals who use smartphone apps daily are no exception. Any tools that try to gauge the success of apps based on these features of the app page alone will likely be ineffective. \\


\bibliographystyle{plain}
\bibliography{finalproject}

\end{document}


%% == Chi's section of examples == %%
% = Includin a Figure = %
%\begin{figure}
%\centering
%\includegraphics[width=1in]{space.jpg} 

%\caption{Another sample figure}
%\label{figure-sample2}
%\end{figure}

% = Generate a bibliography. = %
%\bibliography{proposal}
%\bibliographystyle{unsrt}

%% == Paola's section of examples == %%
%\textcolor{red}{Mandatory section to be included in your final version.}
%~\cite{tohidi06:getting,tohidi06:user}
