%
% CS279 Project Proposal
% Paola Mariselli, Sierra Okolo, Chi Zeng
%

\documentclass{article}

% The geometry package allows for easy page formatting.
\usepackage{geometry}
\geometry{letterpaper}

% Load up special logo commands.
\usepackage{doc}

% Package for formatting URLs.
\usepackage{url}

% For creating our table of contents.
\usepackage{minitoc}

% Packages and definitions for graphics files.
\usepackage{graphicx}
\usepackage{epstopdf}
\DeclareGraphicsRule{.tif}{png}{.png}{`convert #1 `dirname #1`/`basename #1 .tif`.png}

%
% Set the title, author, and date.
%
\title{Sample \LaTeX-Based Research Paper}
\author{Paola Mariselli, Sierra Okolo, Chi Zeng}
\date{}

%
% The document proper.
%
\begin{document}

% Add the title section.
\maketitle

% Add an abstract.
\abstract{
This project aims to determine what matters in determining the appeal of a game or productivity app on the Android market. Adding this test sentence by Paola.
}

\pagebreak
\section{Motivation}

\subsection{What problem will your work address?}

We are attempting to learn how much first impressions might impact the actual adoption of an Android game or productivity application. Mainly, we will examine what factors relevant to first impressions may contribute to the success of such apps. We plan to measure success through popularity (i.e. downloads, ratings). We hope to determine a good model for predicting application successes by examining observable measures for success of mobile applications on the Android Store.

\subsection{Why is it important?}

Currently, app developers take substantial measures to gauge the success of their apps prior to release. Many release a beta version of their app to a restricted community before releasing the app to the public. This beta test allows the developers to not only debug the app, but also obtain an estimate of how much interest their apps will garner. Conducting a meaningful beta test requires a reliable and closed community of trustworthy individuals. Many large firms have this community on hand. For instance, before Google released its iOS application for its Maps service, it first released it to its thousands of employees. Google then got feedback from these employees as well as statistics such as how long individuals stayed on the app and how often they returned. However, small developers do not have such a substantial community of individuals at their disposal for beta testing. Often, these developers just publish their app and monitor metrics over time, adjusting along the way. Our tool would help these developers by offering valuable predictors of app success. \\

Additionally, developers often test specific facets of their app on their own through ads and crowdsourcing services. To test how well an icon performs relative to viable alternatives, app developers sometimes release several different ads on social media sites such as Facebook. Each of these ads would showcase a different candidate icon for an app. A common measure for how well an icon performs in these cases is click-through rate. Similar ideas are often applied to taglines and descriptions of apps. Additionally, services such as Ikonica offer developers the opportunity to send potential icons to a crowd of online individuals and collect their feedback. Our tool would not only formalize such processes that test different facets of an app in a single tool, but also offer a cheaper method for gauging app success through utilizing the common sense of Turkers or potentially other crowds.

\section{Approach}

\subsection{What’s your insight?}

We posit that we can create a way of predicting the success of a game or productivity application without having the developer release an app and monitor metrics in the wild for a duration. We base our tool on the premise that certain factors such as how fun the page of an app appears and how much a viewers trust that page or the app’s icon play a pivotal role in determining the game’s success. \\

Granted, many factors go into the success of game and productivity apps. Games, for instance, must be easily graspable at first, for instance. They must also become harder at a reasonable pace to keep the user engaged. Productivity apps must effectively meet concrete needs. We will be focusing on qualities pertaining to the page of an app, which matters in making strong first impressions.

\subsection{How are you going to go after the problem?}

We plan to leverage the common sense of crowds to gauge these facets and use this data to offer developers an estimate of the success of an app. We plan to develop a statistical model, perhaps based on regression from data gained from the crowd. To obtain data, we plan to scrape the names, icons, and descriptions of free game and productivity apps, so price does not bias one game over another. \\

Afterwards, we plan to develop a questionnaire that asks subjects to rate apps based on the different qualities on a Likert scale. These qualities could include how much subjects trust app pages, whether the pages feel fun, or how clear does the page describe the app’s purposes and functionalities. We plan to gamify this process by challenging subjects to test their entrepreneurial gut. We also plan to have Turkers be subjects. \\

After obtaining this data, we plan to perform a regression to determine how much each quality matters in determining the success of game apps. We then plan to use this regression result as a tool to predict the success of other apps. We also plan to talk to Katharina Reinecke about how else we can interpret our results.

\section{Contribution}

Our goal is to establish the link between the way people feel about given, observable measurements and the application’s actual success, thus delivering a novel approach to predicting the success of mobile applications. We also plan to use this intuition to create a tool that allows users to cheaply predict the success of an application. Beyond software, we hope to determine what factors of the page of a game and productivity application that truly matter in determining the success of the app. This lasting piece of knowledge should benefit developers of games and productivity apps. \\
 
In other words, we intend to uncover the discrete, concrete elements that create subtle psychological influences in users and that propel users to download and use certain  application versus others.  Marketing is a very powerful tool; therefore, through uncovering the words, images, perceptions, and feelings that marketers have historically conveyed through their advertisements in the App Store and dissecting resulting users’ reactions to these subtleties, we can determine the elements that separate a mediocre from a successful marketing campaign. 

\section{Plan}

\subsection{Tuesday, Oct 29: Milestone 1:}

You are required to have a solid draft of a related work section. A good related work section is not just a list of other research --- it should be a thoughtful synthesis that highlights both the strengths and shortcomings of prior work and that contrasts prior work with what you hope to accomplish. \\

\subsubsection{Read recommended papers}
\begin{itemize}
\item “Dive into the literature a little bit to figure out why first impressions matter”
\item B. J. Fogg at Stanford might have done some relevant research
\item Katharina's recent paper is also a good example of how one builds on existing theory to design meaningful empirical investigation (http://iis.seas.harvard.edu/papers/2013/reinecke13aesthetics.pdf)
\item Research how to obtain necessary data
\item Find and read papers related to product first impressions / market research
\item Synthesize related work section from multiple sources
\end{itemize}

\subsection{Tuesday, November 5: Milestone 2:}

You are encouraged to plan your project such that most of the building is done by this date. You should extend your paper by adding an appropriate
technology-related section.

During this phase we plan on designing,  implementing a study, and procuring the data. We will obtain participants from Lab in the Wild. The following highlights the aspects that we will accomplish by this date:

\begin{itemize}
\item Write a script to scrape information on 500 recent Android apps.
\item Design experiment
\item Develop hypotheses
\item Build experiment (potentially on Lab in the Wild)
\item Add dummy data
\item Run small sample study
\item Add pertinent data
\end{itemize}

\subsection{Tuesday, November 12: Milestone 3:}

You should aim to have the meat of the paper written by this date (and have the corresponding work accomplished). The “meat” might, for example, be the section that carefully describes your experiment and your analysis. \\

	By this date we will have a substantial portion of the paper written to include conclusive metrics for mobile apps and a write up regarding how we chose those metrics.  At this time, we will also write up our experimental design, the factors that led us to choose those designs, assumptions we had about our participants, and how that affected the design and implementation of the research. In addition, any necessities to change the design of our study will be appropriately discussed. \\

\subsection{Tuesday, November 19: Milestone 4:}

You’d better had some data by now. The data should be rich enough so that you can reflect on which aspects of your project are working and which aren't. Submit a brief summary of the results and what they mean for your continuing progress. This write up probably won’t make it into the final paper. \\

	We will continue to document trends and observations in our write-up and use that to refine what we really need and how we can best use these metrics to develop a conclusion that addresses whether our hypotheses proved to be right or wrong. \\

\begin{itemize}
\item Plan for data analysis 
\item Conduct data analysis
\item Accept or reject hypotheses
\item Writeup of data analysis and incorporation of results into paper 
\end{itemize}

\subsection{Tuesday, November 26: Milestone 5:}

Your technology should be finalized by now (based on the results from the previous milestone). Revise all sections (intro, related work, description of technology/approach, design of experiment) to reflect the current state of your project. \\

\begin{itemize}
\item Perform a complete review of the paper inspecting for syntax and diction, all the while ensuring that our formulas and tables are accurate and relevant.
\item Complete writeup of data analysis.
\end{itemize}

\section{Related Work}



\end{document}

%% == Examples == %%
% Chi left this section here to provide some examples on how to do various things.

% = Includin a Figure = %
%\begin{figure}
%\centering
%\includegraphics[width=1in]{space.jpg} 
%
%\caption{Another sample figure}
%\label{figure-sample2}
%\end{figure}

% = Generate a bibliography. = %
%\bibliography{latex-sample}
%\bibliographystyle{unsrt}
